\PassOptionsToPackage{unicode=true}{hyperref} % options for packages loaded elsewhere
\PassOptionsToPackage{hyphens}{url}
%
\documentclass[
]{article}
\usepackage{lmodern}
\usepackage{amssymb,amsmath}
\usepackage{ifxetex,ifluatex}
\ifnum 0\ifxetex 1\fi\ifluatex 1\fi=0 % if pdftex
  \usepackage[T1]{fontenc}
  \usepackage[utf8]{inputenc}
  \usepackage{textcomp} % provides euro and other symbols
\else % if luatex or xelatex
  \usepackage{unicode-math}
  \defaultfontfeatures{Scale=MatchLowercase}
  \defaultfontfeatures[\rmfamily]{Ligatures=TeX,Scale=1}
\fi
% use upquote if available, for straight quotes in verbatim environments
\IfFileExists{upquote.sty}{\usepackage{upquote}}{}
\IfFileExists{microtype.sty}{% use microtype if available
  \usepackage[]{microtype}
  \UseMicrotypeSet[protrusion]{basicmath} % disable protrusion for tt fonts
}{}
\makeatletter
\@ifundefined{KOMAClassName}{% if non-KOMA class
  \IfFileExists{parskip.sty}{%
    \usepackage{parskip}
  }{% else
    \setlength{\parindent}{0pt}
    \setlength{\parskip}{6pt plus 2pt minus 1pt}}
}{% if KOMA class
  \KOMAoptions{parskip=half}}
\makeatother
\usepackage{xcolor}
\IfFileExists{xurl.sty}{\usepackage{xurl}}{} % add URL line breaks if available
\IfFileExists{bookmark.sty}{\usepackage{bookmark}}{\usepackage{hyperref}}
\hypersetup{
  pdftitle={Linear Regression Example},
  pdfauthor={Jeff Bostwick},
  pdfborder={0 0 0},
  breaklinks=true}
\urlstyle{same}  % don't use monospace font for urls
\usepackage[margin=1in]{geometry}
\usepackage{color}
\usepackage{fancyvrb}
\newcommand{\VerbBar}{|}
\newcommand{\VERB}{\Verb[commandchars=\\\{\}]}
\DefineVerbatimEnvironment{Highlighting}{Verbatim}{commandchars=\\\{\}}
% Add ',fontsize=\small' for more characters per line
\usepackage{framed}
\definecolor{shadecolor}{RGB}{248,248,248}
\newenvironment{Shaded}{\begin{snugshade}}{\end{snugshade}}
\newcommand{\AlertTok}[1]{\textcolor[rgb]{0.94,0.16,0.16}{#1}}
\newcommand{\AnnotationTok}[1]{\textcolor[rgb]{0.56,0.35,0.01}{\textbf{\textit{#1}}}}
\newcommand{\AttributeTok}[1]{\textcolor[rgb]{0.77,0.63,0.00}{#1}}
\newcommand{\BaseNTok}[1]{\textcolor[rgb]{0.00,0.00,0.81}{#1}}
\newcommand{\BuiltInTok}[1]{#1}
\newcommand{\CharTok}[1]{\textcolor[rgb]{0.31,0.60,0.02}{#1}}
\newcommand{\CommentTok}[1]{\textcolor[rgb]{0.56,0.35,0.01}{\textit{#1}}}
\newcommand{\CommentVarTok}[1]{\textcolor[rgb]{0.56,0.35,0.01}{\textbf{\textit{#1}}}}
\newcommand{\ConstantTok}[1]{\textcolor[rgb]{0.00,0.00,0.00}{#1}}
\newcommand{\ControlFlowTok}[1]{\textcolor[rgb]{0.13,0.29,0.53}{\textbf{#1}}}
\newcommand{\DataTypeTok}[1]{\textcolor[rgb]{0.13,0.29,0.53}{#1}}
\newcommand{\DecValTok}[1]{\textcolor[rgb]{0.00,0.00,0.81}{#1}}
\newcommand{\DocumentationTok}[1]{\textcolor[rgb]{0.56,0.35,0.01}{\textbf{\textit{#1}}}}
\newcommand{\ErrorTok}[1]{\textcolor[rgb]{0.64,0.00,0.00}{\textbf{#1}}}
\newcommand{\ExtensionTok}[1]{#1}
\newcommand{\FloatTok}[1]{\textcolor[rgb]{0.00,0.00,0.81}{#1}}
\newcommand{\FunctionTok}[1]{\textcolor[rgb]{0.00,0.00,0.00}{#1}}
\newcommand{\ImportTok}[1]{#1}
\newcommand{\InformationTok}[1]{\textcolor[rgb]{0.56,0.35,0.01}{\textbf{\textit{#1}}}}
\newcommand{\KeywordTok}[1]{\textcolor[rgb]{0.13,0.29,0.53}{\textbf{#1}}}
\newcommand{\NormalTok}[1]{#1}
\newcommand{\OperatorTok}[1]{\textcolor[rgb]{0.81,0.36,0.00}{\textbf{#1}}}
\newcommand{\OtherTok}[1]{\textcolor[rgb]{0.56,0.35,0.01}{#1}}
\newcommand{\PreprocessorTok}[1]{\textcolor[rgb]{0.56,0.35,0.01}{\textit{#1}}}
\newcommand{\RegionMarkerTok}[1]{#1}
\newcommand{\SpecialCharTok}[1]{\textcolor[rgb]{0.00,0.00,0.00}{#1}}
\newcommand{\SpecialStringTok}[1]{\textcolor[rgb]{0.31,0.60,0.02}{#1}}
\newcommand{\StringTok}[1]{\textcolor[rgb]{0.31,0.60,0.02}{#1}}
\newcommand{\VariableTok}[1]{\textcolor[rgb]{0.00,0.00,0.00}{#1}}
\newcommand{\VerbatimStringTok}[1]{\textcolor[rgb]{0.31,0.60,0.02}{#1}}
\newcommand{\WarningTok}[1]{\textcolor[rgb]{0.56,0.35,0.01}{\textbf{\textit{#1}}}}
\usepackage{graphicx,grffile}
\makeatletter
\def\maxwidth{\ifdim\Gin@nat@width>\linewidth\linewidth\else\Gin@nat@width\fi}
\def\maxheight{\ifdim\Gin@nat@height>\textheight\textheight\else\Gin@nat@height\fi}
\makeatother
% Scale images if necessary, so that they will not overflow the page
% margins by default, and it is still possible to overwrite the defaults
% using explicit options in \includegraphics[width, height, ...]{}
\setkeys{Gin}{width=\maxwidth,height=\maxheight,keepaspectratio}
\setlength{\emergencystretch}{3em}  % prevent overfull lines
\providecommand{\tightlist}{%
  \setlength{\itemsep}{0pt}\setlength{\parskip}{0pt}}
\setcounter{secnumdepth}{-2}
% Redefines (sub)paragraphs to behave more like sections
\ifx\paragraph\undefined\else
  \let\oldparagraph\paragraph
  \renewcommand{\paragraph}[1]{\oldparagraph{#1}\mbox{}}
\fi
\ifx\subparagraph\undefined\else
  \let\oldsubparagraph\subparagraph
  \renewcommand{\subparagraph}[1]{\oldsubparagraph{#1}\mbox{}}
\fi

% set default figure placement to htbp
\makeatletter
\def\fps@figure{htbp}
\makeatother


\title{Linear Regression Example}
\author{Jeff Bostwick}
\date{11/10/2020}

\begin{document}
\maketitle

{
\setcounter{tocdepth}{2}
\tableofcontents
}
\hypertarget{load-packages-and-explore-data}{%
\subsection{Load Packages and Explore
Data}\label{load-packages-and-explore-data}}

Loading Packages

\begin{Shaded}
\begin{Highlighting}[]
\KeywordTok{library}\NormalTok{(sjPlot)}
\KeywordTok{library}\NormalTok{(dplyr)}
\KeywordTok{library}\NormalTok{(sjlabelled)}
\KeywordTok{library}\NormalTok{(sjmisc)}
\KeywordTok{library}\NormalTok{(ggplot2)}
\KeywordTok{theme_set}\NormalTok{(}\KeywordTok{theme_sjplot}\NormalTok{())}
\end{Highlighting}
\end{Shaded}

Loading Dataset

Previewing the Dataset

\begin{Shaded}
\begin{Highlighting}[]
\KeywordTok{head}\NormalTok{(data)}
\end{Highlighting}
\end{Shaded}

\begin{verbatim}
##   Make      Model Year            Engine.Fuel.Type Engine.HP Engine.Cylinders
## 1  BMW 1 Series M 2011 premium unleaded (required)       335                6
## 2  BMW   1 Series 2011 premium unleaded (required)       300                6
## 3  BMW   1 Series 2011 premium unleaded (required)       300                6
## 4  BMW   1 Series 2011 premium unleaded (required)       230                6
## 5  BMW   1 Series 2011 premium unleaded (required)       230                6
## 6  BMW   1 Series 2012 premium unleaded (required)       230                6
##   Transmission.Type    Driven_Wheels Number.of.Doors
## 1            MANUAL rear wheel drive               2
## 2            MANUAL rear wheel drive               2
## 3            MANUAL rear wheel drive               2
## 4            MANUAL rear wheel drive               2
## 5            MANUAL rear wheel drive               2
## 6            MANUAL rear wheel drive               2
##                         Market.Category Vehicle.Size Vehicle.Style highway.MPG
## 1 Factory Tuner,Luxury,High-Performance      Compact         Coupe          26
## 2                    Luxury,Performance      Compact   Convertible          28
## 3               Luxury,High-Performance      Compact         Coupe          28
## 4                    Luxury,Performance      Compact         Coupe          28
## 5                                Luxury      Compact   Convertible          28
## 6                    Luxury,Performance      Compact         Coupe          28
##   city.mpg Popularity  MSRP
## 1       19       3916 46135
## 2       19       3916 40650
## 3       20       3916 36350
## 4       18       3916 29450
## 5       18       3916 34500
## 6       18       3916 31200
\end{verbatim}

Viewing the Structure of the Dataset

\begin{Shaded}
\begin{Highlighting}[]
\KeywordTok{str}\NormalTok{(data)}
\end{Highlighting}
\end{Shaded}

\begin{verbatim}
## 'data.frame':    11914 obs. of  16 variables:
##  $ Make             : chr  "BMW" "BMW" "BMW" "BMW" ...
##  $ Model            : chr  "1 Series M" "1 Series" "1 Series" "1 Series" ...
##  $ Year             : int  2011 2011 2011 2011 2011 2012 2012 2012 2012 2013 ...
##  $ Engine.Fuel.Type : chr  "premium unleaded (required)" "premium unleaded (required)" "premium unleaded (required)" "premium unleaded (required)" ...
##  $ Engine.HP        : int  335 300 300 230 230 230 300 300 230 230 ...
##  $ Engine.Cylinders : int  6 6 6 6 6 6 6 6 6 6 ...
##  $ Transmission.Type: chr  "MANUAL" "MANUAL" "MANUAL" "MANUAL" ...
##  $ Driven_Wheels    : chr  "rear wheel drive" "rear wheel drive" "rear wheel drive" "rear wheel drive" ...
##  $ Number.of.Doors  : int  2 2 2 2 2 2 2 2 2 2 ...
##  $ Market.Category  : chr  "Factory Tuner,Luxury,High-Performance" "Luxury,Performance" "Luxury,High-Performance" "Luxury,Performance" ...
##  $ Vehicle.Size     : chr  "Compact" "Compact" "Compact" "Compact" ...
##  $ Vehicle.Style    : chr  "Coupe" "Convertible" "Coupe" "Coupe" ...
##  $ highway.MPG      : int  26 28 28 28 28 28 26 28 28 27 ...
##  $ city.mpg         : int  19 19 20 18 18 18 17 20 18 18 ...
##  $ Popularity       : int  3916 3916 3916 3916 3916 3916 3916 3916 3916 3916 ...
##  $ MSRP             : int  46135 40650 36350 29450 34500 31200 44100 39300 36900 37200 ...
\end{verbatim}

Summary of the Dataset

\begin{Shaded}
\begin{Highlighting}[]
\KeywordTok{summary}\NormalTok{(data)}
\end{Highlighting}
\end{Shaded}

\begin{verbatim}
##      Make              Model                Year      Engine.Fuel.Type  
##  Length:11914       Length:11914       Min.   :1990   Length:11914      
##  Class :character   Class :character   1st Qu.:2007   Class :character  
##  Mode  :character   Mode  :character   Median :2015   Mode  :character  
##                                        Mean   :2010                     
##                                        3rd Qu.:2016                     
##                                        Max.   :2017                     
##                                                                         
##    Engine.HP      Engine.Cylinders Transmission.Type  Driven_Wheels     
##  Min.   :  55.0   Min.   : 0.000   Length:11914       Length:11914      
##  1st Qu.: 170.0   1st Qu.: 4.000   Class :character   Class :character  
##  Median : 227.0   Median : 6.000   Mode  :character   Mode  :character  
##  Mean   : 249.4   Mean   : 5.629                                        
##  3rd Qu.: 300.0   3rd Qu.: 6.000                                        
##  Max.   :1001.0   Max.   :16.000                                        
##  NA's   :69       NA's   :30                                            
##  Number.of.Doors Market.Category    Vehicle.Size       Vehicle.Style     
##  Min.   :2.000   Length:11914       Length:11914       Length:11914      
##  1st Qu.:2.000   Class :character   Class :character   Class :character  
##  Median :4.000   Mode  :character   Mode  :character   Mode  :character  
##  Mean   :3.436                                                           
##  3rd Qu.:4.000                                                           
##  Max.   :4.000                                                           
##  NA's   :6                                                               
##   highway.MPG        city.mpg        Popularity        MSRP        
##  Min.   : 12.00   Min.   :  7.00   Min.   :   2   Min.   :   2000  
##  1st Qu.: 22.00   1st Qu.: 16.00   1st Qu.: 549   1st Qu.:  21000  
##  Median : 26.00   Median : 18.00   Median :1385   Median :  29995  
##  Mean   : 26.64   Mean   : 19.73   Mean   :1555   Mean   :  40595  
##  3rd Qu.: 30.00   3rd Qu.: 22.00   3rd Qu.:2009   3rd Qu.:  42231  
##  Max.   :354.00   Max.   :137.00   Max.   :5657   Max.   :2065902  
## 
\end{verbatim}

\hypertarget{task-2---clean-your-dataset}{%
\subsection{Task 2 - Clean your
dataset}\label{task-2---clean-your-dataset}}

Capture all columns that are character fields

\begin{Shaded}
\begin{Highlighting}[]
\NormalTok{cols <-}\StringTok{ }\KeywordTok{names}\NormalTok{(data)[}\KeywordTok{vapply}\NormalTok{(data, is.character, }\KeywordTok{logical}\NormalTok{(}\DecValTok{1}\NormalTok{))]}
\NormalTok{data[,cols] <-}\StringTok{ }\KeywordTok{lapply}\NormalTok{(data[,cols],trimws)}
\end{Highlighting}
\end{Shaded}

Convert missing values to NAs

\begin{Shaded}
\begin{Highlighting}[]
\NormalTok{data[data}\OperatorTok{==}\StringTok{"N/A"}\NormalTok{] =}\StringTok{ }\OtherTok{NA}
\end{Highlighting}
\end{Shaded}

Use sapply(), which is like a for loop that goes through each column of
the dataset and applys the function to it

\begin{Shaded}
\begin{Highlighting}[]
\KeywordTok{sapply}\NormalTok{(data, }\ControlFlowTok{function}\NormalTok{(x) }\KeywordTok{mean}\NormalTok{(}\KeywordTok{is.na}\NormalTok{(x)))}
\end{Highlighting}
\end{Shaded}

\begin{verbatim}
##              Make             Model              Year  Engine.Fuel.Type 
##      0.0000000000      0.0000000000      0.0000000000      0.0000000000 
##         Engine.HP  Engine.Cylinders Transmission.Type     Driven_Wheels 
##      0.0057915058      0.0025180460      0.0000000000      0.0000000000 
##   Number.of.Doors   Market.Category      Vehicle.Size     Vehicle.Style 
##      0.0005036092      0.3140842706      0.0000000000      0.0000000000 
##       highway.MPG          city.mpg        Popularity              MSRP 
##      0.0000000000      0.0000000000      0.0000000000      0.0000000000
\end{verbatim}

It looks like Market.Category column has a high number of missing values
(roughly 31.4\%) It might be smart to remove this column from the
dataset by running the code below

\begin{Shaded}
\begin{Highlighting}[]
\NormalTok{data}\OperatorTok{$}\NormalTok{Market.Category <-}\StringTok{ }\OtherTok{NULL}
\end{Highlighting}
\end{Shaded}

Return only observations that have no missing values and preview the
dataset

\begin{Shaded}
\begin{Highlighting}[]
\NormalTok{data <-}\StringTok{ }\NormalTok{data[}\KeywordTok{complete.cases}\NormalTok{(data),]}
\KeywordTok{head}\NormalTok{(data)}
\end{Highlighting}
\end{Shaded}

\begin{verbatim}
##   Make      Model Year            Engine.Fuel.Type Engine.HP Engine.Cylinders
## 1  BMW 1 Series M 2011 premium unleaded (required)       335                6
## 2  BMW   1 Series 2011 premium unleaded (required)       300                6
## 3  BMW   1 Series 2011 premium unleaded (required)       300                6
## 4  BMW   1 Series 2011 premium unleaded (required)       230                6
## 5  BMW   1 Series 2011 premium unleaded (required)       230                6
## 6  BMW   1 Series 2012 premium unleaded (required)       230                6
##   Transmission.Type    Driven_Wheels Number.of.Doors Vehicle.Size Vehicle.Style
## 1            MANUAL rear wheel drive               2      Compact         Coupe
## 2            MANUAL rear wheel drive               2      Compact   Convertible
## 3            MANUAL rear wheel drive               2      Compact         Coupe
## 4            MANUAL rear wheel drive               2      Compact         Coupe
## 5            MANUAL rear wheel drive               2      Compact   Convertible
## 6            MANUAL rear wheel drive               2      Compact         Coupe
##   highway.MPG city.mpg Popularity  MSRP
## 1          26       19       3916 46135
## 2          28       19       3916 40650
## 3          28       20       3916 36350
## 4          28       18       3916 29450
## 5          28       18       3916 34500
## 6          28       18       3916 31200
\end{verbatim}

\hypertarget{task-3---split-into-training-and-test-set}{%
\subsection{Task 3 - Split into training and test
set}\label{task-3---split-into-training-and-test-set}}

Now we will only be selecting numeric columns from the dataset for our
linear regression model

\begin{Shaded}
\begin{Highlighting}[]
\NormalTok{data_num <-}\StringTok{ }\NormalTok{data }\OperatorTok\StringTok{ }\KeywordTok{select_if}\NormalTok{(is.numeric)}
\end{Highlighting}
\end{Shaded}

The target variable for our machine learning model is the price column
(MSRP) Now we will create a histogram to see the MSRP distribution

\begin{Shaded}
\begin{Highlighting}[]
\KeywordTok{hist}\NormalTok{(data_num}\OperatorTok{$}\NormalTok{MSRP, }\DataTypeTok{breaks=}\DecValTok{100}\NormalTok{)}
\end{Highlighting}
\end{Shaded}

\includegraphics{Linear-Regression-Example_files/figure-latex/unnamed-chunk-7-1.pdf}

This shows us that there are some outliers in our column as the majority
of cars have the price in this region. these outliers can cause issues
with the model, so we can filter the dataset to include cars with a
price range of 15,000 and 50,000.

\begin{Shaded}
\begin{Highlighting}[]
\NormalTok{data_num <-}\StringTok{ }\NormalTok{data_num }\OperatorTok\StringTok{ }\KeywordTok{filter}\NormalTok{(MSRP }\OperatorTok{>}\StringTok{ }\DecValTok{15000}\NormalTok{) }\OperatorTok\StringTok{ }\KeywordTok{filter}\NormalTok{(MSRP }\OperatorTok{<}\StringTok{ }\DecValTok{50000}\NormalTok{)}
\end{Highlighting}
\end{Shaded}

Now we will split our dataset into a training and test set. To get
consistent results, and to make sure the partitions are reproducable,
the seed will need to be set to any integer. We will now select 80\% of
the dataset to training and remaining 20\% will be the test dataset. To
do this, we will get the number of rows that will account to 80\%, and
then use the floor() function to round up to the next integer.

\begin{Shaded}
\begin{Highlighting}[]
\KeywordTok{set.seed}\NormalTok{(}\DecValTok{123}\NormalTok{)}
\NormalTok{size <-}\StringTok{ }\KeywordTok{floor}\NormalTok{(}\FloatTok{0.8} \OperatorTok{*}\StringTok{ }\KeywordTok{nrow}\NormalTok{(data_num))}
\end{Highlighting}
\end{Shaded}

Now we will use the sample() function to randomly select 80\% of rows
from your dataset and store the row numbers.

\begin{Shaded}
\begin{Highlighting}[]
\NormalTok{train_ind <-}\StringTok{ }\KeywordTok{sample}\NormalTok{(}\KeywordTok{seq_len}\NormalTok{(}\KeywordTok{nrow}\NormalTok{(data_num)), }\DataTypeTok{size =}\NormalTok{ size)}
\end{Highlighting}
\end{Shaded}

To get the training dataset set you can filter the dataset to include
the row numbers, to get the test dataset you can filter the dataset to
ignore the row numbers.

\begin{Shaded}
\begin{Highlighting}[]
\NormalTok{train <-}\StringTok{ }\NormalTok{data_num[train_ind, ]}
\NormalTok{test <-}\StringTok{ }\NormalTok{data_num[}\OperatorTok{-}\NormalTok{train_ind, ]}
\end{Highlighting}
\end{Shaded}

\hypertarget{task-4---fit-linear-regression-model-and-interpret-model-summary-statistics}{%
\subsection{Task 4 - Fit linear regression model and interpret model
summary
statistics}\label{task-4---fit-linear-regression-model-and-interpret-model-summary-statistics}}

A linear regression model is a model that assumes a linear relationship
between the predictors and the response variable.

This means that the response variable can be calculated from a linear
combination of predictors.

In this dataset the the response variable is the MSRP column, while the
remaininag columns are the predictors.

The goals is to build a model to predict the MSRP column, by using the
characteristcs such as Engine, HP, number of doors, etc.

To build the linear regression model we will use the lm() function and
focus on the model equation and the dataset to be used.

The model equations can be returned in column names while the dataset we
will be using will be the training dataset.

\begin{Shaded}
\begin{Highlighting}[]
\NormalTok{model <-}\StringTok{ }\KeywordTok{lm}\NormalTok{(MSRP }\OperatorTok{~}\StringTok{ }\NormalTok{.,}\DataTypeTok{data =}\NormalTok{ train)}
\KeywordTok{summary}\NormalTok{(model)}
\end{Highlighting}
\end{Shaded}

\begin{verbatim}
## 
## Call:
## lm(formula = MSRP ~ ., data = train)
## 
## Residuals:
##      Min       1Q   Median       3Q      Max 
## -18494.4  -3819.5   -717.9   3407.0  20580.3 
## 
## Coefficients:
##                    Estimate Std. Error t value Pr(>|t|)    
## (Intercept)      -2.784e+05  4.012e+04  -6.940 4.32e-12 ***
## Year              1.430e+02  2.006e+01   7.128 1.13e-12 ***
## Engine.HP         1.121e+02  1.779e+00  63.006  < 2e-16 ***
## Engine.Cylinders -1.144e+03  9.981e+01 -11.461  < 2e-16 ***
## Number.of.Doors   3.567e+02  9.030e+01   3.950 7.91e-05 ***
## highway.MPG      -9.825e+01  2.898e+01  -3.390 0.000703 ***
## city.mpg          1.638e+02  2.544e+01   6.440 1.29e-10 ***
## Popularity       -2.279e-01  4.962e-02  -4.593 4.45e-06 ***
## ---
## Signif. codes:  0 '***' 0.001 '**' 0.01 '*' 0.05 '.' 0.1 ' ' 1
## 
## Residual standard error: 5538 on 6384 degrees of freedom
## Multiple R-squared:  0.5929, Adjusted R-squared:  0.5925 
## F-statistic:  1328 on 7 and 6384 DF,  p-value: < 2.2e-16
\end{verbatim}

In the Coefficients we can see the statistical significance. Predictive
variables that are significantly associated with the outcome variables
are marked with stars.

The higher the amount of stars, the more significant the predictors are.
For a given predictor value, the coefficient (which is called the
estimate) can be interepted as the average effect on the response
varable of one unit increase in predictor given that all other
predictors are fixed.

For example, if the engine HP increases by 1 unit and all other
predictors are kept constant, then the price of the car will increase by
111 units.

The Residual Standard Error, Mult R-Squared, and F-Statistics are
metrics that are used to check how well the model fits your data.

Residual Standard Error - corresponds to the prediction error in the
training set and represents roughly the average difference between the
observed values and the predicted values of teh model.\\
In this model the Residual Standard Error is 5495. That mean on average
you can expect a diviation of 5495 in the price prediction. The
R-Squared rangeds from 0 to 1 and represents the proportion of teh
variation and the response variable that can be explained by the model
predictor variables.\\
The higher the R-Squared value, the better the model is. However, a
problem with the R squared is that it will always increase as more
predictors are added to teh model. even if the predictors are only
weakly associated with the outcome of the respnose variable. A solution
is to adjust the R squared value by taking into the account the number
of predictive variables. the adjestment in the adjusteed R-squared value
in the summary output is the correction for the number of predictive
variables included in the model. So you mainly consider the adjusted R
squared value your value it .59 (which is good). F-statistic gives the
overall significance of the model. it assess whether at least one
predict value has a non-zero coefficient. the p value of less than
2.22-16 shows that the model is highly siginificant

\begin{Shaded}
\begin{Highlighting}[]
\KeywordTok{plot_model}\NormalTok{(model, }\DataTypeTok{show.values =} \OtherTok{TRUE}\NormalTok{, }\DataTypeTok{value.offset =} \FloatTok{0.2}\NormalTok{)}
\end{Highlighting}
\end{Shaded}

\includegraphics{Linear-Regression-Example_files/figure-latex/unnamed-chunk-13-1.pdf}

this plot shows the coefficient and significance value now you can build
a linear regression model by sepcifying the predictors that you want.
for examples you might want to include only 3 predictors instead all of
them in you dataset

\begin{Shaded}
\begin{Highlighting}[]
\NormalTok{model2 <-}\StringTok{ }\KeywordTok{lm}\NormalTok{(MSRP }\OperatorTok{~}\StringTok{ }\NormalTok{Engine.HP }\OperatorTok{+}\StringTok{ }\NormalTok{highway.MPG }\OperatorTok{+}\StringTok{ }\NormalTok{Engine.Cylinders, }\DataTypeTok{data =}\NormalTok{ train)}
\end{Highlighting}
\end{Shaded}

\hypertarget{task-5---plot-and-analyse-model-residuals}{%
\subsection{Task 5 - Plot and analyse model
residuals}\label{task-5---plot-and-analyse-model-residuals}}

residuals can show how poorly a matter represents data. They are left
over values of teh response variable after fitting a model to data, and
they can reveil unexplained patterns in the data by the fitted model.
Using this information, not only could we check if linear regression
assumptions are met, but you could improve your model as well.

\begin{Shaded}
\begin{Highlighting}[]
\KeywordTok{par}\NormalTok{(}\DataTypeTok{mfrow=}\KeywordTok{c}\NormalTok{(}\DecValTok{2}\NormalTok{,}\DecValTok{2}\NormalTok{))}
\KeywordTok{plot}\NormalTok{(model)}
\end{Highlighting}
\end{Shaded}

\includegraphics{Linear-Regression-Example_files/figure-latex/unnamed-chunk-15-1.pdf}
this wil set the plotting pane to include 4 plots( 2 rows and 2 columns)

R vs Fitted show if residuals have nnn-linear patterns. fitted values
are on the x axis and the residuals (which are how far the fitted values
are from the observed values)are on the y axis. there could be a
non-linear relationship between predictive variables and the response
variables and teh pattern could show up in this plot. if the model does
not capture the non-linear relationship. if you find equally residuals
around the horizontal line, without any distinct patterns, that is a
good indication that you dont have non-linear relationships. in this
plot we do not see any any distinctive patterns. Normal QQ plot which
shows if residuals are naormally distributed. Do The residuals follow a
straight line? or do they deviate severly? It is good when residuals are
lined well on the straight line. but in reality you will see some
deviations. In this plot we do not se e much deviation until towards the
end where some data points are deiviated.

The scale location plot show if residualdes are diparred equally along
the ranges of predictors. this is how you can check the assumption of
equal variance. It is good if you see a horizonalt line withh equally
diparred points. in this model the residuals appear to be randomlly
spread.

residuals vs leverage plot helps you to find influential cases in your
dataset.\\
These cases could be extremem cases against the regression line, and
could alter results if you exclude them from you model. In this graph
pattens are not relevant. You should watch out for outlier values in the
upper right and lower right corners. those parts are the places where
the cases can be influential against the regression line. look for cases
outside of the dash line (the cook distance). cases that are outside the
cook distance (meaning they are high cook distance codes) cases are
influental to teh regression results. in our model we can see that
observation 6519 and 6522 are far beyond the cook distance lines. there
are influential cases and will alter your model if you remove them. )

the 4 plots show potential problematic cases, with the row number of the
data in your dataset. if some cases are noted in each of the plots, you
might wan tto take a closer look at them. Is there anything sepcial with
those points? or was there an error in teh data entry? you can go back
to the building step and you can try include/excluding predictors and
see if the diagnostic plots improve

\hypertarget{task-6---predict-future-values-and-calculate-model-error-metrics}{%
\subsection{Task 6 - Predict future values and calculate model error
metrics}\label{task-6---predict-future-values-and-calculate-model-error-metrics}}

you will predict the MSRP value of the test dataset and compare it to
the with the observed MSRP values. you can youse the predict functions
with parameters model and test data.

\begin{Shaded}
\begin{Highlighting}[]
\NormalTok{test}\OperatorTok{$}\NormalTok{pred <-}\StringTok{ }\KeywordTok{predict}\NormalTok{(model, }\DataTypeTok{newdata =}\NormalTok{ test)}
\end{Highlighting}
\end{Shaded}

now you can plot the predicted and observed values

\begin{Shaded}
\begin{Highlighting}[]
\KeywordTok{par}\NormalTok{(}\DataTypeTok{mfrow=}\KeywordTok{c}\NormalTok{(}\DecValTok{1}\NormalTok{,}\DecValTok{1}\NormalTok{))}
\KeywordTok{ggplot}\NormalTok{(test, }\KeywordTok{aes}\NormalTok{(}\DataTypeTok{x=}\NormalTok{MSRP, }\DataTypeTok{y=}\NormalTok{pred))}\OperatorTok{+}
\StringTok{  }\KeywordTok{geom_point}\NormalTok{()}\OperatorTok{+}
\StringTok{  }\KeywordTok{geom_smooth}\NormalTok{(}\DataTypeTok{method=}\StringTok{"lm"}\NormalTok{, }\DataTypeTok{color=}\StringTok{"blue"}\NormalTok{)}\OperatorTok{+}
\StringTok{  }\KeywordTok{theme_bw}\NormalTok{()}
\end{Highlighting}
\end{Shaded}

\begin{verbatim}
## `geom_smooth()` using formula 'y ~ x'
\end{verbatim}

\includegraphics{Linear-Regression-Example_files/figure-latex/unnamed-chunk-17-1.pdf}

on teh x axis you can see the observered MSRP values and on the Y you
can see the predictive values. Now we will calculate the error metrics
of the linear model.. we will first find the error. which is observed
value subtraced from the respective predicted value.

\begin{Shaded}
\begin{Highlighting}[]
\NormalTok{error <-}\StringTok{ }\NormalTok{test}\OperatorTok{$}\NormalTok{pred }\OperatorTok{-}\StringTok{ }\NormalTok{test}\OperatorTok{$}\NormalTok{MSRP}
\end{Highlighting}
\end{Shaded}

we will calc 2 error metrics RMSE is a good measure to see how accurate
the model predicts a response. this is a good test for fit, if the main
purpose of the model is prediction.

\begin{Shaded}
\begin{Highlighting}[]
\NormalTok{rmse <-}\StringTok{ }\KeywordTok{sqrt}\NormalTok{(}\KeywordTok{mean}\NormalTok{(error}\OperatorTok{^}\DecValTok{2}\NormalTok{))}
\NormalTok{rmse}
\end{Highlighting}
\end{Shaded}

\begin{verbatim}
## [1] 5371.526
\end{verbatim}

the rmse value is 5546, which is fine since the range of the rsme value
is between 15k and 50k the second metric is MAE (mean absolute error).
measures the avg magnitue of the errors in your predictions predictions
without considering their direction

\begin{Shaded}
\begin{Highlighting}[]
\NormalTok{mae <-}\StringTok{ }\KeywordTok{mean}\NormalTok{(}\KeywordTok{abs}\NormalTok{(error))}
\NormalTok{mae}
\end{Highlighting}
\end{Shaded}

\begin{verbatim}
## [1] 4268.753
\end{verbatim}

the mae is 4401. this means that on avg you would expect an error
magnitude of 4401 in your predictions. This error can be either positive
or negative.

in rmse, since the errors are squared before they are averaged, the rmse
give a relativly high weight to large errors. this means the rse should
be more usedful when large errors a particularly undesirable. but from
an interpretataion stnadpoint mean absolte error is better.

\end{document}
